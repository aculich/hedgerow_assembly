\documentclass[12pt]{letter}
\usepackage[margin=2.5cm]{geometry}
\usepackage{mathpazo}
\usepackage[pdftex]{graphicx}
\usepackage{color}
\usepackage[normalem]{ulem}
\usepackage{times}
\usepackage{textcomp}

\address{Dept of Environmental Science,\\
  Policy, and Management\\
  130 Mulford Hall\\
  University of California, Berkeley\\
  Berkeley, California, 94720, USA} \raggedright
\begin{document}

\begin{letter}{}

  \opening{Dear editor,}

  Please find attached the manuscript \textit{Major interaction% I like the "we are pleased to submit our manuscript" format...
    reorganizations punctuate the assembly of pollination networks}
  which we are submitting for consideration as a Letter in
  \textit{Ecology Letters}.

Our research deals with two fundamental aspects of ecological theory: to understand not only how species-rich communities assemble, but also how these assemblages change through time. Furthermore, as the
  world continues to lose species at an alarming rate, it has become
  increasingly imperative to aid the recovery of lost interactions and component biodiversity through ecological restoration. When a species goes extinct, not only a species is lost, but also its interactions. We know little, however, about how to re-assemble interacting communities through restoration, or the process of ecological network assembly
  more generally. 

  Using extensive surveys of pollinators spanning eight years
  comprising \texttildelow $20,000$ pollinator visitation records, we
  explore the assembly of plant-pollinator communities at native plant restorations in the Central Valley of California. For the first time in the ecological litterature we employ a newly
  developed method to examine the temporal changes in networks. Among other things, we find that species are highly dynamic in their network position,
  causing community assembly to be punctuated by major interaction
  reorganizations. The most persistent and generalized species are
  also the most variable in their network positions, contrary to what
  is expected through preferential attachment theory --- an assembly
  theory otherwise well-supported in the network literature. Our study
  is the first long-term study on the temporal assembly of ecological
  networks. It also challenges the hypothesis that mutualistic systems
  assemble through preferential attachment (Bascompte and Stouffer,
  2009).

%I think this is an important paragraph - we have to be white males!!! I am not sure about the three fields....
Our results are compelling and provide empirical evidence that widen our understading on how communities assembly and how species interactions changes through time. Furthermore, our results also contribute to the knowledge of how communities will be able to maintain function in the face of species extinctionAnd finally, our results challenge the view that communities assemble trhough preferential attachment. We believe that these exciting results that link three major ecological fields (interaction networks, community dinamics and restauration ecology) that will be of broad interest to the readership of \textit{Ecology Letters}. 

  Our manuscript is original and was carried out fully by the authors.
  All authors agree with the contents of the manuscript.  This
  manuscript is not published, nor is it in consideration for
  publication elsewhere.  All research not of the authors' is fully
  acknowledged.  The authors declare no conflict of interest. All
  appropriate ethical standards were followed.  Thank you for
  reviewing our manuscript and we hope you will find it suitable for
  publication.
  
  Regards,
  Lauren C. Ponisio, PhD\\
  Claire Kremen, Professor
%is there a reason why I am not here? Just curious =0) 
\end{letter}
\end{document}

%%% Local Variables:
%%% mode: latex
%%% TeX-PDF-mode: t
%%% End:
